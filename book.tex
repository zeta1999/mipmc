\documentclass[a4paper, 12pt]{book}

\usepackage{hyperref}

\begin{document}

\pagenumbering{roman}
\title{The Malevolent Ingenuity of Preprocessor Metaprogramming in C}
\author{Temirkhan Myrzamadi}
\date{\today}
\maketitle

\tableofcontents

\newpage

\section{Foreword}

The macro system is probably the most contradictory feature of C: it provides a way to
\href{https://en.wikipedia.org/wiki/Extensible_programming}{extend} the host programming
language with new syntactical constructs, thus forcing some ill-formed programs fail
to compile, through is an increasingly treacherous technology, being able to potentially
cause a numerous amount of bugs out of nothing.

This book was derived by my continuous experiments with the preprocessor: I have implemented
some of the concepts of functional programming, including
\href{https://en.wikipedia.org/wiki/Algebraic_data_type}{algebraic data types (ADTs)} with
\href{https://en.wikipedia.org/wiki/Proof_by_exhaustion}{exhaustive}
\href{https://en.wikipedia.org/wiki/Pattern_matching}{pattern matching},
\href{https://en.wikipedia.org/wiki/Parametric_polymorphism}{parametric polymorphism},
\href{https://en.wikipedia.org/wiki/Kind_(type_theory)}{higher-kinded types}, and
\href{https://en.wikipedia.org/wiki/Generalized_algebraic_data_type}{generalised ADTs}, as
well as from the world of object orientation: classes,
\href{https://en.wikipedia.org/wiki/Dynamic_dispatch}{dynamic dispatch}, and all the coherent
stuff: \href{https://en.wikipedia.org/wiki/Type_introspection}{type introspection}, type-safe
error handling facilities and more.

So what I want to say: writing macros is an art. Macros, overall, can even be considered as
computer languages on their own, with custom syntax, given by their signatures, and
semantics, defined by their bodies. In this book, I have accomplished my best concerning
the development of macros, and I do believe that it will advance a reader to a completely
new dimension of metaprogramming, show how much more beautiful macro interfaces can be, and
develop the intuition in such a tricky topic.

\newpage

\section{Preface}


\end{document}
