\documentclass[a4paper, 12pt]{book}
\usepackage{hyperref} 
\begin{document}

\pagenumbering{roman}
\title{The Malevolent Ingenuity of Preprocessor Metaprogramming in C}
\author{Temirkhan Myrzamadi}
\date{\today}
\maketitle

\tableofcontents

\newpage

\section{Foreword}

The macro system is probably the most contradictory feature of C: it provides a way to extend
the host programming language with new syntactical constructs, thus forcing some ill-formed programs
fail to compile, through is an increasingly treacherous technology, being able to potentially
cause a numerous amount of bugs out of nothing.

This book was derived by my continuous experiments with the preprocessor: I have implemented some of
the concepts of functional programming, including algebraic data types (ADTs) with exhaustive
pattern matching, as well as from the world of object orientation: classes, dynamic dispatch, and
all the coherent stuff: parametric polymorphism, higher-kinded types, type introspection, type-safe
error handling facilities and more.

So what I want to say: writing macros is an art. Macros, overall, can be considered as computer
languages on their own, with custom syntax, given by their signatures, and semantics, defined by
their bodies. I believe that this book will be useful for everyone programming in C, directing a
reader to much cleaner macro interfaces and their implementations.

\newpage

\section{Preface}


\end{document}
